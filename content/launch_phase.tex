\newpage
\section{Launch Phase}
\subsection{General Analysis}
	\subsubsection{Requirements	 Analysis}
		\begin{itemize}
			\item Functional Requirements ("system shall do 'requirement' ")
			\item Non-functional Requirements ("system shall be 'requirement' ")
			\item Behavioural Requirements ("how the system shall react")
			\item Performance Requirements ("how well does it have to be done")
		\end{itemize}
		% Functional Table
		\begin{table}[h!]
		\begin{tabular} [b] {| c |  p{3cm} | c | p{10cm} |}
			\hline
			\textbf{ID} & \textbf{Requirement} & \textbf{Reference} & \textbf{Description} \\\hline
			F-1 & Communication &  &  \\ \hline
			F-1.1 & Web Server &  & HTTP communication between the hub and the web server; Web-services are going to be use. \\ \hline
			F-1.2 & System &  & System uses Power Line Communication to send and receive data between the modules and the hub. \\ \hline
			F-1.2.1 & Protocol &  & Not Defined yet... \\ \hline
			F-2 & Routing &  &  \\ \hline
			F-2.1 & Storage &  & Energy stored is routed to the consumers. \\ \hline
			F-2.2 & Producers &  & Energy being produced is routed to the storage devices and if full directly routed to the consumers. \\ \hline
			% &~  &~  &~  \\ \hline
			% &~  &~  &~  \\  \hline
			% &~  &~  &~  \\  \hline
			% &~  &~  &~  \\  \hline
			% &~  &~  &~  \\  \hline
			% &~  &~  &~  \\  \hline
			% &~  &~  &~  \\  \hline
			% &~  &~  &~  \\  \hline
		\end{tabular}
		\caption{Functional: System shall do...}
		\end{table}
		
		% Non-Function Table
		\begin{table}[h!]
			\begin{tabular} [b] {| c |  p{3cm} | c | p{10cm} |}
			\hline
			\textbf{ID} & \textbf{Requirement} & \textbf{Reference} & \textbf{Description} \\\hline
			NF-1 & User Interface &  &  \\ \hline
			NF-1.1 & Web Interface &  & User Friendly; Easy to understand so everyone without energy knowledge could use the system. \\ \hline
			NF-1.2 & HW Interface &  & Simple; User Friendly; Only available for the user of the system. \\ \hline
			NF-2 & Electrical &  &  \\ \hline
			NF-2.1 & Voltage &  & The system work in a range 27.5V +-  10\%.\\ \hline
			NF-2.2 & Current &  & Not defined yet... But very important \\ \hline
			% &~  &~  &~  \\ \hline
			% &~  &~  &~  \\  \hline
			% &~  &~  &~  \\  \hline
			% &~  &~  &~  \\  \hline
			% &~  &~  &~  \\  \hline
			% &~  &~  &~  \\  \hline
			% &~  &~  &~  \\  \hline
			% &~  &~  &~  \\  \hline
		\end{tabular}
		\caption{Non-Function: System shall be...}
		\end{table}
		\newpage
		% Behavioural Table
		\begin{table}[h!]
			\begin{tabular} [b] {| c |  p{3cm} | c | p{10cm} |}
			\hline
			\textbf{ID} & \textbf{Requirement} & \textbf{Reference} & \textbf{Description} \\\hline
			B-1 & Status &  &  \\\hline
			B-1.1 & Webpage &  & Warning will be posted in the webpage interface. \\\hline
			B-1.2 & Physical &  & LEDs will show the current status of the connected device Running / Stand By / Warning. \\\hline
			B-1.3 & Report &  & An email will be sent to the defined user when an error occurs. \\\hline
			B-2 & Energy Control &  &  \\\hline
			B-2.1 & Over-production &  & Change devices status to stand by. \\\hline
			B-2.2 & Under-production &  & Use of the energy store in the input/output module and power grid if it have to be a fast charging. \\\hline
			B-2.3 & Normal-production &  & The energy will be routed from the input devices to the output devices. \\\hline
			B-3 & Disaster Avoid &  & Humidity and Temperature sensor will be placed inside the system housing. \\\hline
			B-3.1 & Humidity &  & "When the humidity is high, stop system." \\\hline
			B-3.2 & Temperature &  & "When temperature is high,  stop system." \\\hline
			% &~  &~  &~  \\\hline
			% &~  &~  &~  \\  \hline
			% &~  &~  &~  \\  \hline
			% &~  &~  &~  \\  \hline
			% &~  &~  &~  \\  \hline
			% &~  &~  &~  \\  \hline
			% &~  &~  &~  \\  \hline
			% &~  &~  &~  \\  \hline
		\end{tabular}
		\caption{Behavioural: How the system shall react...}
		\end{table}
		
		% Perfomance Table
		\begin{table}[h!]
			\begin{tabular} [b] {| c | p{3cm} | c | p{10cm} |}
			\hline
			\textbf{ID} & \textbf{Requirement} & \textbf{Reference} & \textbf{Description} \\ \hline
			P-1 & Hardware &  &  \\ \hline
			P-1.1 & Housing &  & The housing have to be water-proof and with a good heat dissipation. \\ \hline
			% &~  &~  &~  \\  \hline
			% &~  &~  &~  \\  \hline
			% &~  &~  &~  \\  \hline
			% &~  &~  &~  \\  \hline
			% &~  &~  &~  \\  \hline
			% &~  &~  &~  \\  \hline
			% &~  &~  &~  \\  \hline
			% &~  &~  &~  \\  \hline
		\end{tabular}
		\caption{Performance: How well does it have to be done...}
		\end{table}
		\newpage
	\subsubsection{Problem Domain Analysis}
			\paragraph{Block Diagram:}
				\textbf{Candidates in the system}
				\\\textbf{Consumers:} Units that consumes power (e.g. light, washing machine, electric car)
				\\\textbf{Producers:} Modules that delivers energy to the hub (e.g. Wind-turbine, photovoltaic-cells)
				\\\textbf{Storage:} Modules that both consumes when the system over produces and "produces" (gives energy to the consumers) when needed. 
				\\\textbf{Temperature:} Unit that measures the outdoor temperature. If the temperature is too high or low it might damage the hardware.
				\\\textbf{Humidity:} Unit that measures the humidity where the system is places. If the humidity is too high.
				\newline
			\paragraph{Block Diagram:}
				\textbf{Event in the system}
				\\\textbf{TemperatureBelowLimit}
				\\\textbf{TemperatureAboveLimit}
				\\\textbf{HumidityAboveLimit}
				\\\textbf{EnergyAboveLimit}
				\\\textbf{EnergyBelowLimit}
				\\\textbf{InitModule}
				\\\textbf{StartModule}
				\\\textbf{StopModule}
				\\\textbf{StandbyModule}
				\\
				\begin{table}[h!]
					\begin{tabular}{| r | c | c | c | c | c |}
					\hline
					~ & Consumers & Producers & Storage & Temp & Humidity \\ \hline
					TempBellowLimit & ~ & ~ & ~ & X & ~ \\ \hline
					TempAboveLimit & ~ & ~ & ~ & X & ~ \\ \hline
					HumidityAboveLimit & ~ & ~ & ~ & ~ & X \\ \hline
					InitModule & X & X & X & ~ & ~ \\ \hline
					StartModule & X & X & X & ~ & ~ \\ \hline
					StopModule & X & X & X & ~ & ~ \\ \hline
					StandbyModule & ~ & X & X & ~ & ~ \\ \hline
					EnergyAboveLimit & ~ & X & ~ & ~ & ~ \\ \hline
					EnergyBelowLimit & ~ & X & ~ & ~ & ~ \\ \hline
					% &~  &~  &~  \\  \hline
					% &~  &~  &~  \\  \hline
					% &~  &~  &~  \\  \hline
					% &~  &~  &~  \\  \hline
					% &~  &~  &~  \\  \hline
					% &~  &~  &~  \\  \hline
					% &~  &~  &~  \\  \hline
					% &~  &~  &~  \\  \hline
					\end{tabular}
				\end{table}
				\\ Text about all different states in the system.  \textbf{DENNES}
				\\\textbf{Table with all the above candidates and event combined:}
				\newline
			\paragraph{State Machine Diagrams}
			 \textbf{DENNES}
	\subsubsection{Usage Domain Analysis}
		\begin{itemize}
			\item Use Cases - Put in diagrams of user + system (modules)
		\end{itemize}
	\subsubsection{Interface Analysis}
			Put in the comic book story!  \textbf{JESUS}
		\begin{itemize}
			\item User Interface Descriptions
				
			\item System Interface Descriptions
		\end{itemize}
	\subsubsection{Function Analysis}
	 \textbf{THIS}
		\begin{itemize}
			\item Recognise the direction of the device ( input, output, both ).
			\item Start / Stop slave devices from web interface or physical module ( Control over the connected modules ).
			\item Routes energy from the input to output ports.
			\item Control power status of slave modules.
			\item Log device data ( uptime of the port and power amount since connected ).
			\item More to add here...
		\end{itemize}
	\subsubsection{System Dynamics}
			Put in example from eudp: %eudp.dk/index.php/Sequence_Diagrams
			 \textbf{JESUS / DENNES}
		\begin{itemize}
			\item Communication Diagram
			\item Sequence Diagram
		\end{itemize}
	\subsubsection{General Analysis Specification}
		 \textbf{THIS}
\subsection{General Architecture Design}
	\subsubsection{Design Criteria}
		Write in what parts should be used (mostly off the shelf things).  \textbf{THIS}
	\subsubsection{Subsystem Design}
		\begin{itemize}
			\item Subsystem Architecture Diagram
			\item Detailed Block Diagrams
		\end{itemize}
	\subsubsection{Architecture Dynamics}
		Hubs communication with submodules
		\begin{itemize}
			\item Sequence Diagrams
		\end{itemize}
	\subsubsection{Architectural Specification and Constraints}
\subsection{Technical Platform}
	A lot of technical stuffs. 
	\subsubsection{Hardware Specifications}
	\subsubsection{Software Specifications}